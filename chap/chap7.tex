
\chapter{总结与展望}
在文論本中,我們建議並實施一個名為BRTMS的區塊鏈實名交易監督系統,不僅為分別花錢和賺取加密貨幣的客戶和商家,同時也能協助府金融監督單位審計加密貨幣交易籌集稅收。 此外,加密貨幣交易實驗的初步結果,使用比特幣測試幣及於章節4.1所設計的BRTMS數據庫,我們以著名的比特幣加密貨幣錢包實作的Java應用程序和Android應用程序客戶端。
	
此外,修改其在Android系統上的開放原始碼應用程式,使得本論文闡述之概念能夠實際在區塊鏈上運行,以期未來加密貨幣不僅能維持目前的便利性,還可以讓使用者不用擔心交易後找不到賣家,使一般民眾能更安心使用加密貨幣作為日常生活的行動支付管道。

	\paragraph{提出的BRTMS架構還包括以下特色}

		\begin{enumerate}
			\item 向建議的BRTMS進行商業註冊需經政府批准。
			\item 所有出售的商品將受到政府審查。
			\item 消費者仍然匿名以確保個人信息的隱私。
			\item 消費者的交易記錄不能被刪除。
			\item 如果消費者有關交易上訴的問題,他們需要提交交易發票或證明付款人或收款人地址的訪問權。
			\item 所有交易記錄均公開透明。
			\item 原始交易數據採用區塊鏈技術記錄,具有高度的可靠性,分散和未篡改的數據。
			\item 政府可以檢查建議的BRTMS系統的交易記錄,以快速有效的方式審查稅務信息。
		\end{enumerate}

	\section{提議的BRTMS的優點總結}

		\paragraph{消費者}交易信息公開透明,保護消費者的權益。由於交易是可信的,並有明確的時間戳。當消費者需要上交他們的消費者權利時,他們可以從提出的系統中獲得更有效和可信的證據。
		\paragraph{商家}企業可以根據所有數字化交易信息為自己的業務目的進行統計和計算。這可以減少手動操作計算結果中的錯誤。統計數據甚至可以與商店的庫存管理相結合,使貨物和資金達到更加便利地進行統計,進一步改善業務的會計準確性和人工成本。
		\paragraph{政府}在解決交易糾紛的過程中,可以提供更多可信的證據供參考。數字交易收據也可以解決紙本收據丟失或破損的問題。考慮到稅收問題,政府可以審查具有高信譽的商業交易細節作為稅收計算程序,以定制稅標準,減少許多稅收糾紛。
	