% Copyright (c) 2014,2016 Casper Ti. Vector
% Public domain.

\chapter{結論和未來的工作}
在本文中,我們建議並實施一個名為BPCSS的基於區塊鏈的支付收款監督系統,不僅為分別花錢和賺取數字貨幣的客戶和商家,而且還為政府金融監督單位審計數字貨幣交易,然後幫助籌集 稅收。 此外,數字貨幣交易實驗的初步結果,使用著名的Testnet比特幣和雲數據庫以及我們實現的Java應用程序和Android應用程序的著名普及數字錢包,證明了數字貨幣交易中保留的成本效益以及擬議的付款人和收款人之間的監督BPCSS。
	
	\section{提出的BPCSS架構還包括以下特色}

		\paragraph{一、}向建議的BPCSS進行商業註冊需經政府批准。
		\paragraph{二、}所有出售的商品將受到政府審查。
		\paragraph{三、}消費者仍然匿名以確保個人信息的隱私。
		\paragraph{四、}消費者的交易記錄不能被刪除。
		\paragraph{五、}如果消費者有關交易上訴的問題,他們需要提交交易發票或證明付款人或收款人地址的訪問權。
		\paragraph{六、}所有交易記錄均公開透明。
		\paragraph{七、}原始交易數據採用區塊鏈技術記錄,具有高度的可靠性,分散和未篡改的數據。
		\paragraph{八、}政府可以檢查建議的BPCSS系統的交易記錄,以經濟有效的方式審查稅務信息。

	\section{提議的BPCSS的優點總結}

		\paragraph{對消費者}交易信息公開透明,保護消費者的權益。由於交易是可信的,並有明確的時間戳。當消費者需要上交他們的消費者權利時,他們可以從提出的系統中獲得更有效和可信的證據。
		\paragraph{業務}企業可以根據所有數字化交易信息為自己的業務目的進行統計和計算。這可以減少手動操作計算結果中的錯誤。統計數據甚至可以與商店的庫存管理相結合,使貨物和資金達到所需的餘額,進一步改善業務的會計準確性和人工成本。
		\paragraph{政府}:在解決交易糾紛的過程中,可以提供更多可信的證據供參考。數字交易收據也可以解決偽造或丟失紙質收據的問題。考慮到稅收問題,政府可以審查具有高信譽的商業交易細節作為稅收計算程序,以找出征稅標準,減少許多稅收糾紛。

在不久的將來,我們將為政府財政監督部門提供監督職能。特別是,我們需要考慮一種具有成本效益的方式來增強訪問客戶,商店和金融監管部門的雲數據庫時的安全和隱私保護。
% vim:ts=4:sw=4
