   
\chapter{总结与展望}
本文设计并实现一个名为BRTMS(Bitcoin Realtime Transaction Monitoring System)的比特币的实时交易监督系统,其工作包括背景调研比特币技术上的优势与劣势、匿名交易衍生出无法保障顾客权益、政府无法课征税收及存在着透过比特币洗钱的问题。为解决上述问题,在技术调研方面针对比特币地址的相关算法、地址的生成过程、多重签章算法以及区块链技术进行剖析。设计系统之前得进行详细的需求分析以及交易模型分析。设计BRTMS时,加入Government Green Address使该系统成为实时监督系统,并实现BTMS与BRTMS的客户端与服务器端,最后对BTMS(Bitcoin Transaction Monitoring System)进行功能性测试和性能测试。性能测试的成果中可知BRTMS比BTMS的交易完成时间从平均287.12秒缩减至1.55秒,达到更好的顾客消费体验。

在BRTMS中,顾客交易信息是匿名且公开透明,同时保障顾客消费权益;商家可以根据所有数字化交易信息为自己的业务目的进行统计和计算,减少手动操作计算结果中的错误。统计数据甚至可以与商家的库存管理相结合,使货物和资金更加便利地进行统计,进一步改善信息的准确性和降低人工成本;政府在解决交易纠纷的过程中拥有更多可信的证据供参考。数字交易凭据也可以解决纸本交易凭据丢失或破损的问题。本文提出的BRTMS架构包括以下特色:
		\begin{enumerate}
			\item 引入匿名支付给实名的交易模型至加密货币交易系统。
			\item 设计与实现于加密货币中匿名支付给实名的交易监督系统。
			\item 于政府端实现多重签章算法,使实时的交易监督比Green Address节点更快速。
			\item 透过多重签章算法使得商家可以预防双重支付攻击。
			\item 商家商品管理让商家可进行库存管理及商家商品管理。
			\item 实现于加密货币中,消费者匿名同时也让消费者拥有消费者权益。
			\item 借由将商家实名,BRTMS使政府主管机关可以有效获得税收。
		\end{enumerate}



		
BRTMS不仅兼顾顾客和商家同时协助府金融监督单位审计加密货币交易筹集税收。本系统中以bitcoinj库实作的Java应用程序和Android应用程序客户端,修改其在Android系统上的开放源代码应用程序,使得本论文阐述之概念能够实际在区块链上运行,未来加密货币不仅能维持目前的便利性,还可以让用户不用担心交易后找不到卖家,使一般民众能更安心使用加密货币作为日常生活的移动支付管道。
