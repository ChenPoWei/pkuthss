 
\chapter{总结与展望}
在本論文中,設計並實現一個名為BRTMS的比特幣的實時交易監督系統,其工作包括背景調研現今的比特區塊鏈技術上的優勢與劣勢,比特幣匿名交易衍生出無法保障顧客權益、政府無法課徵稅收以及存在著透過比特幣洗錢的問題,還有在各國對比特幣資產流通的監管政策與態度。為了能夠深入的解決上述問題,在技術調研方面針對比特幣地址的相關算法、比特幣地址的生成過程以、多重簽章算法以及區塊鏈技術進行深度的剖析。為了解決背景調研中存在的諸多問題,便須設計BTMS,設計系統之前得進行詳細的需求分析以及交易模型分系,交易模型分析的過程。設計BRTMS時,加入Government Green Address使該系統成為實時監督系統,並實現BTMS與BRTMS的客戶端與服務器端,最後對BTMS進行功能性測試和性能測試。在性能測試的成果中可知BRTMS比BTMS的交易完成時間更少,可以達到更好的顧客消費體驗。

在BRTMS不僅兼顧顧客和商家同時也能協助府金融監督單位審計加密貨幣交易籌集稅收。本會系統中的以著名的比特幣加密貨幣錢包實作的Java應用程序和Android應用程序客戶端,修改其在Android系統上的開放原始碼應用程式,使得本論文闡述之概念能夠實際在區塊鏈上運行,未來加密貨幣不僅能維持目前的便利性,還可以讓使用者不用擔心交易後找不到賣家,使一般民眾能更安心使用加密貨幣作為日常生活的行動支付管道。本文提出的BRTMS架構還包括以下特色如下:

		\begin{enumerate}
			\item 引入匿名支付給實名的交易模型至加密貨幣交易系統。
			\item 設計與實現於加密貨幣中匿名支付給實名的交易監督系統。
			\item 於政府端實現多重簽章算法,使實時的交易監督比Green Address節點更快速。
			\item 透過多重簽章算法使得商家可以預防雙重支付攻擊。
			\item 商家商品管理讓商家可進行庫存管理及商家商品管理。
			\item 實現於加密貨幣中,消費者匿名同時也讓消費者擁有消費者權益。
			\item 藉由將商家實名,BRTMS使政府主管機關可以有效獲得稅收。
		\end{enumerate}

在BRTMS中,顧客匿名交易的信息公開透明,保護顧客的權益。由於交易是可信的,並有明確的時間戳。當顧客需要上交他們的顧客權利時,他們可以從提出的系統中獲得更有效和可信的證據;商家可以根據所有數字化交易信息為自己的業務目的進行統計和計算。這可以減少手動操作計算結果中的錯誤。統計數據甚至可以與商家的庫存管理相結合,使貨物和資金達到更加便利地進行統計,進一步改善業務的會計準確性和人工成本;政府在解決交易糾紛的過程中,可以提供更多可信的證據供參考。數字交易收據也可以解決紙本收據丟失或破損的問題。考慮到稅收問題,政府可以審查具有高信譽的商業交易細節作為稅收計算程序,以定制稅標準,減少許多稅收糾紛。
	