% Copyright (c) 2014,2016 Casper Ti. Vector
% Public domain.

\specialchap{序言}
%\pkuthssffaq % 中文测试文字。
現金法定貨幣,收據及交易數據庫存在著一些缺點。如現金很難杜絕假鈔的橫行,收據有著偽造的可能,在交易數據庫中資料不一致,數據庫被DDOS攻擊,交易數據被竄改,數據庫損毀,也都是在交易過程中曾出現的窘境。

於2009年加密貨幣 - 比特幣的問世,以密碼學、網路學、貨幣銀行學為基礎創建了新一代的網路貨幣。竄改、公開交易數據檢視、使用者匿名性、自動運作不須人為運營的多項特性。至今區塊鏈技術已成為IBM、摩根大通、微軟、谷歌、英特爾重點開發項目,被視為改善銀行運作效率、降低運營成本、提升資訊安全、建立公開數據的最佳方法。為解決現金、收益及交易數據庫存在之問題,本論文採用以區塊為基礎的數字貨幣比特幣為貨幣,進行商業化收銀系統開發。不僅僅是比特幣算法穩定、交易公開透明、不可被竄改的特性外,更是本論文加入監督標籤,使得在匿名交易轉為部分實名交易,促使監管部門能有更好的貨幣技術提升,亦可建立自動化的稅務審查機制,大幅降低人事成本,亦可提高交易系統的信息可靠度及穩定度。
% vim:ts=4:sw=4
