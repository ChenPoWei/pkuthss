% Copyright (c) 2014,2016 Casper Ti. Vector
% Public domain.

\chapter{致谢}
%\pkuthssffaq % 中文测试文字。
原本就對比特幣區塊鏈技術深感興趣的我,來到了北京大學攻讀工程碩士學位。在這段期間深怕著會因為科系的關係而影響到了我的研究方向,在導師雙選了劉京老師,老師相當支持我做自己的研究,後來也見到了李傑教授,大力鼓勵者我繼續往科研的方向前進,老師的宏亮的聲音、霸氣的指導深植我心。段莉華老師也相當支持我做學術研究,因為段老師也一度的前往北京大學的校本部探討密碼學的研究。

在台灣實習的我來到了台灣最高學術研究機構中央研究院資訊科學所繼續展開我的科研路,同時也延續著之前與銘傳大學王家輝老師合作的科技部計畫“比特幣監督收銀系統”,因為有著計畫的補助也使得在求學的路上較無經濟壓力,也因為計劃上的補助,使我能夠順利地前往義大利羅馬參加IEEE WiMob會議發表論文“Blockchain-based payment collection supervision system using pervasive Bitcoin digital wallet.”,參加了台灣最大的計算機會議TANET發表論文“匿名加密貨幣與實名商家交易的有效行動支付監督平台之建置與實作-以比特幣為例”,也得到了TANET會議的最佳論文獎,也要對與我合作的最佳夥伴江柏憲同學,我們共創了大學時期專題研究的第一名,這次我們也一舉奪下了TANET的最佳論文,相信都在我們的人生道路中寫下了嶄新的一頁。感謝李開輝教授願意教導我並讓我在其旗下做科學研究,同時給予我最大的資源與協助,並總是指引我研究方向。

除了在諸位教授的諄諄教誨中,使我有機會完成這篇論文外,也要誠摯的感謝於二零一四年帶我認識比特幣的啟蒙老師楊哲豪先生,沒有他沒有現在多達三萬四千人的比特幣中文社團的社群,也不會給予我有這樣的機會了解比特幣的運作原理,更不會有現在的台灣比特幣產業鏈,奠定區塊鏈在台灣的望能奠定區塊鏈科技產業發展之基石。
