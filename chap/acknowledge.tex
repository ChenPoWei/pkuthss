% Copyright (c) 2014,2016 Casper Ti. Vect
 
\chapter{致谢}
%\pkuthssffaq % 中文测试文本。
首先得感谢李杰教授大力鼓励者我继续往科研的方向前进,同时给予我最多的论文资源与多次的论文指导,总是指引我研究方向,老师严谨的治学态度使我终生受益。刘京老师对比特币区块链技术的了解,在我寻求问题解答的过程中给予不同的见解。

在实习单位中央研究院信息科学所继续展开我的科研路,同时也延续着之前与铭传大学王家辉老师合作的科技部计划“比特币监督收银系统”,因为有着计划的补助也使得在求学的路上获得更充足的预算,也因为计划上的补助,使我能够顺利地前往意大利罗马参加IEEE WiMob会议发表论文"Blockchain-based payment collection supervision system using pervasive Bitcoin digital wallet.",参加台湾最大的计算机会议TANET 发表论文“匿名加密货币与实名商家交易的有效行动支付监督平台之建置与实作-以比特币为例”\supercite{tanet},也得到了TANET 会议的最佳论文奖,在我人生道路中写下了崭新的一页。感谢李开晖教授愿意教导我并让我在其旗下做科学研究。