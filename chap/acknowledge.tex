% Copyright (c) 2014,2016 Casper Ti. Vect

\chapter{致谢}
%\pkuthssffaq % 中文测试文字。
首先得感謝李傑教授大力鼓勵者我繼續往科研的方向前進,同時給予我最多的論文資源與多次的論文指導,總是指引我研究方向,老師嚴謹的治學態度使我終生受益。劉京老師對比特幣區塊鏈技術的支持,在尋求問題解答的過程中,給予不同的見解加以討論。

在實習單位中央研究院資訊科學所繼續展開我的科研路,同時也延續著之前與銘傳大學王家輝老師合作的科技部計畫“比特幣監督收銀系統”,因為有著計畫的補助也使得在求學的路上獲得更充足的預算,也因為計劃上的補助,使我能夠順利地前往義大利羅馬參加IEEE WiMob會議發表論文“Blockchain-based payment collection supervision system using pervasive Bitcoin digital wallet.”\supercite{Blockchain-basedpaymentcollectionsupervisionsystemusingpervasiveBitcoindigitalwallet},參加台灣最大的計算機會議TANET發表論文“匿名加密貨幣與實名商家交易的有效行動支付監督平台之建置與實作-以比特幣為例”,也得到了TANET會議的最佳論文獎,在我人生道路中寫下了嶄新的一頁。感謝李開暉教授願意教導我並讓我在其旗下做科學研究。