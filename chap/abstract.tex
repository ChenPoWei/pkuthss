% Copyright (c) 2014,2016 Casper Ti. Vector
% 摘要
\begin{cabstract}

	比特幣是一個集成網絡學、密碼學、貨幣銀行學的加密貨幣,加密貨幣市場中,有數以千計的貨幣種類在市場流動著。值得一提的是,至今比特幣的點對點式電子現金系統還未出現過錯誤。比特幣最大的特色在於去中心化與匿名化,以去中心化的基礎建構出一個其他人無法管控的點對點金流,但也因為其匿名之特性,存在著三項問題,分別為無法追蹤、無法得到稅收以及在交易的過程中無法保障消費者權益。第一,在比特幣系統中,沒有任何一個使用者可以要求每一個人落實實名制,政府主管機關與相關人士難以追查每⼀筆資金的真正持有者,進而增加洗錢防制的困難性。第二,稅收更是國家經濟基礎運作的資金來源,現今的國家並無支持以比特幣交易相關的收銀系統或是制定出相關的稅務標準,使得政府無法從加密貨幣這方面的金融交易獲得稅收。第三,現有的比特幣交易模型皆為匿名與匿名之間的交易,並無開立收據,無法保障消費者權益。

	本論文設計與實現一個区块链匿名加密貨幣為主的实名交易监督系统,論文工作包括五項。第一,交易模型分析:分析五種交易模型,發現匿名支付給實名交易,達到保障消費者權益、保護消費者隱私以及使政府可以課徵稅收。第二,系統設計:以匿名支付給實名交易為基礎設計支持加密貨幣的系統架構。第三,系統實現:使用Java編程語言,實現主要系統、商店和商品信息管理子系統、移動裝置收款及交易子系統以及客戶端移動支付和交易子系統。第四,系統優化:
	解決區塊鏈速度緩慢,引入多重簽章算法構建實時監督系統。第五,功能與性能測試:實現系統後,對本系統進行功能測試,且對原始監督系統與實時監督系統進行性能測試與分析。

	本論文貢獻如下:
	\begin{enumerate}
		\item 引入匿名支付給實名的交易模型至加密貨幣系統。
		\item 設計與實現於加密貨幣中匿名支付給實名的交易監督系統。
		\item 導入多重簽章算法,設計與實現實時的交易監督系統,提升交易速度。
		\item 實現於加密貨幣中,消費者匿名同時也讓消費者擁有消費者權益。
		\item 藉由將商家實名,使政府主管機關可以獲得稅收的有效加密貨幣交易監督模型。
	\end{enumerate}


\end{cabstract}

\begin{eabstract}
	Financial technology is booming Today, the blockchain technology is also a key development project.The most famous blockchain technology is nothing more than an article by Bitcoin by Satoshi Nakamoto in 2009: A Peer-to-Peer E-Cash System Paper.He laid the foundation for blockchain technology.Bitcoin is a cryptocurrency that integrates network science, cryptography, and currency banking.In the cryptocurrency market, there are thousands of cryptocurrencies flowing in markets.It is worth mentioning that, since Bitcoin point to point electronic cash system has not experienced an error.

	The biggest feature of Bitcoin is decentralization and anonymization.Bitcoin is based on a paradigm shift to create point-to-point traffic that no one can control.However, because of its anonymity, it is difficult for the relevant government officials to trace the true holder of each sum of money.In the traditional central bank transnational transfer needs basic real-name verification.Real-name system can effectively filter out the occurrence of money laundering.But in Bitcoin's point-to-point e-cash system, no single user can ask everyone to implement a real-name system.Thereby increasing the difficulty of money-laundering prevention and control.Apart from the hard-to-trace characteristics, taxation is the source of funding for the operation of the government.Present-day countries do not support bitcoin-related cashier systems or set out the tax standards.Making the government unable to get the tax in this area.


	Again by the above cannot manage the flow of funds, cannot be tracked, cannot get tax three starting point. This thesis is devoted to designing a "blockchain real-name transaction monitoring system". Before designing the system, several trading models were also explored. Found that there are cash payments to pay anonymous anonymous, anonymous payment to real-name model, in the credit card payment has a real name payment to the real name, real name paid to anonymous and then paid to the real name, the above four models. Through the analysis of the above model can be learned. The rise of the awareness of personal privacy, only paid by the anonymous real name, we can do without revealing consumer information, consumer protection can also be done to protect rights and interests. In peer-to-peer electronic cash, it still stays in the anonymous mode of payment anonymously. This essay aims to design a supervisory cashier system for anonymous real-name cryptocurrencies. To practice the consumer anonymity, but also allow consumers to have the rights and interests of the transaction model. 

\end{eabstract}

% vim:ts=4:sw=4
