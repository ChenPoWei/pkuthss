% Copyright (c) 2014,2016 Casper Ti. Vector

\begin{cabstract}

	⽐特幣是⼀個集成網絡學、密碼學、貨幣銀⾏學並以區塊鏈為基礎的加密貨幣。加密貨幣市場中有數以千計的貨幣種類在市場流動著,有別於其他加密貨幣的,是⽐特幣特有的去中⼼化與匿名化。區塊鏈技術雖能保有交易信息的透明性與交易數據的不可變動性,但也因為其匿名之特性存在著三項問題。第一,在⽐特幣交易系統中,沒有任何⼀個使⽤者可以要求每⼀個⼈落實實名制,因此,政府主管機關與相關⼈⼠難以追查每⼀筆資⾦的真正持有者,進⽽擴大流動資金的不透明性。其次,稅收是國家經濟基礎運作的資⾦來源,現今的國家並無⽀持以⽐特幣交易相關的⽀付系統或是制定出相關的稅務標準,使得政府無法從加密貨幣這⽅⾯的⾦融交易獲得稅收。第三,現有的⽐特幣交易模型皆為匿名與匿名之間的交易,並無開⽴收據,亦無法保障消費者權益。

	本論⽂設計與實現⼀個比特幣的實時交易監督系統-BRTMS,論文工作包括分析五種交易模型,發現由匿名顧客支付給實名商家的交易監督可同時實踐保障消費者權益、保護顧客隱私以及使政府可以課徵稅收。由需求分析將系統模塊劃分為用戶註冊與登入、產品管理、職工管理、商家交易管理,以及顧客交易管理五種模塊,同時採用Java編程語言和比特幣開源錢包,以五種模塊為基礎實現主要系統,以及商家和商品信息管理、移動裝置收款及交易、客戶端移動支付和交易三個子系統,並且為解決區塊鏈速度緩慢問題引入多重簽章算法構建實時監督系統。實現系統後,對本系統進行功能測試,且對原始監督系統與有多重簽章演算法支援的實時監督系統進行性能測試與分析。

	比特幣通過本系統,當顧客使用移動裝置客⼾端擁有有詳細的交易明細,同時保障消費者權益;藉由商家和商品信息管理子系統,讓商家可以更容易進⾏商品管理及庫存管理;在政府端實現多重簽章算法,讓商家可預防雙重⽀付攻擊,且大幅減少平均交易所需時間,也能使政府主管機關有效獲得稅收。



\end{cabstract}

\begin{eabstract}
	Bitcoin is a cryptocurrency that integrates networking, cryptography, and currency banking. In the cryptocurrency market, there are thousands of different currencies that flow in the market. It is worth mentioning that Bitcoin's peer-to-peer e-cash system has not made mistakes so far. The biggest feature of Bitcoin is decentralization and anonymization, and the decentralized foundation creates a peer-to-peer currency flow that others cannot control. However, because of its anonymity, there are three issues: the inability to track the flow of funds, the inability to receive taxes, and the inability to protect consumer rights in the course of transactions. First, in the Bitcoin system, no one can require every user to realize the real name system, and it is difficult for the government authorities and related person to trace the true holder of each fund, thereby increasing the difficulty of money-laundering prevention. Second, taxation is a source of funding for the operation of the country’s economic base. Today’s countries do not support the use of Bitcoin transactions related to the cash register system or formulating relevant tax standards, so making it impossible for the government to obtain tax revenues from cryptocurrency financial transactions. Third, the existing Bitcoin transaction models are all transactions between anonymous and anonymous transactions, and no receipts have been issued to protect consumer rights.

	This thesis designs and implements a real-time real-name transaction monitoring system based on anonymous blockchain cryptocurrency. The work of the thesis includes five parts. First, transaction model analysis: analyzing five transaction models and discovering anonymous payments to real-name transactions to protect consumer rights, protect consumer privacy, and enable government to collect taxes. Second, the system design: designing a real-time cryptocurrency transaction monitoring system architecture to support anonymous payments to real-name transactions. Third, the system is implemented: using the Java programming language and opensource Bitcoin wallet to implement the main system including store and merchandise information management subsystem, mobile device collection and payment subsystem, and client mobile payment and transaction subsystem. Fourth, system optimization: to solve the problem of slow blockchain transaction speed, the multiple signature algorithms is introduced to build a real-time transaction monitoring system. Fifth, functional and performance tests: after the system is implemented, functional tests are conducted on the implemented system, and performance testing and analysis are performed on the original supervisory system without multiple signature algorithm and the real-time supervisory system.
	The contributions of this paper are as follows:Introduce anonymous payment to real-name trading models for the transaction cryptocurrency system for cryptocurrencies;Design and implementation of anonymous payments to real-name transaction monitoring systems; The multiple signatures algorithm is implemented on government site, the transaction speed is faster than using GreenAddress web site for proposed real-time transaction monitoring system; For cryptocurrency, consumer anonymity also allows consumers to have consumer rights; By putting the real name on the merchandise store, the government authorities can obtain an effective cryptocurrency transaction supervision model for taxation.

\end{eabstract}

% vim:ts=4:sw=4
