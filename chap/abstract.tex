% Copyright (c) 2014,2016 Casper Ti. Vector
% 摘要
\begin{cabstract}

	金融科技蓬勃發展的今天,區塊鏈技術也是重點發展對象。區塊鏈技術最著名的不外乎是於2009年Satoshi Nakamoto提出的一篇名為比特幣:一種點對點式的電子現金系統論文\supercite{bitcoinpaper},奠定了區塊鏈技術的開始。比特幣是一個集成網絡學、密碼學、貨幣銀行學的加密貨幣,加密貨幣市場中,有數以千計的貨幣種類在市場流動著。值得一提的是,至今比特幣點對點式的電子現金系統還未出現過錯誤。

	比特幣最大的特色在於去中心化與匿名化,以去中⼼化的基礎建構出⼀個其他人無法管控的點對點⾦流,但也因為其匿名之特性,使得政府相關人⼠難以追查每⼀筆資⾦的真正持有者,在傳統的中心化銀行跨國轉帳中都需要基本的實名制驗證,藉由實名制有效過濾洗錢的發生。但在比特幣點對點的電子現金系統中,沒有任何一個使用者可以要求每一個人落實實名制,進而增加洗錢防制的困難性。除了難以追蹤的特點外,稅收更是國家基礎運作的資⾦來源,現今的國家並無支持比特幣相關的收銀系統或是制定出相關的稅務標準,也使得政府無法從這方面獲得稅收。

	由上述無法管理資金流、無法追蹤、無法得到稅收三項出發點。本論文致力於設計一個区块链的实名交易监督系统。在設計該系統前,也探討了多種模式下的交易模型,發現現金支付已經存在匿名支付給匿名、匿名支付給實名的模型,在刷卡支付中有著實名支付給實名、實名支付給匿名再支付給實名,上述的四種模型。經由上述模式的分析,可以得知,個人隱私的意識崛起,唯由匿名支付給實名時,才可以做到不透露消費者信息,亦可做到保障消費者權益。在點對點的電子現金市場中,還是停留在匿名支付給匿名的模式中,本論⽂致⼒於設計出匿名⽀付實名的加密貨幣市場之監督收銀系統,以實踐消費者匿名,同時也讓消費者擁有費者權益的交易模型。
\end{cabstract}

\begin{eabstract}
	Financial technology is booming Today, the blockchain technology is also a key development project.The most famous blockchain technology is nothing more than an article by Bitcoin by Satoshi Nakamoto in 2009: A Peer-to-Peer E-Cash System Paper.He laid the foundation for blockchain technology.Bitcoin is a cryptocurrency that integrates network science, cryptography, and currency banking.In the cryptocurrency market, there are thousands of cryptocurrencies flowing in markets.It is worth mentioning that, since Bitcoin point to point electronic cash system has not experienced an error.

	The biggest feature of Bitcoin is decentralization and anonymization.Bitcoin is based on a paradigm shift to create point-to-point traffic that no one can control.However, because of its anonymity, it is difficult for the relevant government officials to trace the true holder of each sum of money.In the traditional central bank transnational transfer needs basic real-name verification.Real-name system can effectively filter out the occurrence of money laundering.But in Bitcoin's point-to-point e-cash system, no single user can ask everyone to implement a real-name system.Thereby increasing the difficulty of money-laundering prevention and control.Apart from the hard-to-trace characteristics, taxation is the source of funding for the operation of the government.Present-day countries do not support bitcoin-related cashier systems or set out the tax standards.Making the government unable to get the tax in this area.


	Again by the above cannot manage the flow of funds, cannot be tracked, cannot get tax three starting point. This thesis is devoted to designing a "blockchain real-name transaction monitoring system". Before designing the system, several trading models were also explored. Found that there are cash payments to pay anonymous anonymous, anonymous payment to real-name model, in the credit card payment has a real name payment to the real name, real name paid to anonymous and then paid to the real name, the above four models. Through the analysis of the above model can be learned. The rise of the awareness of personal privacy, only paid by the anonymous real name, we can do without revealing consumer information, consumer protection can also be done to protect rights and interests. In peer-to-peer electronic cash, it still stays in the anonymous mode of payment anonymously. This essay aims to design a supervisory cashier system for anonymous real-name cryptocurrencies. To practice the consumer anonymity, but also allow consumers to have the rights and interests of the transaction model. 

\end{eabstract}

% vim:ts=4:sw=4
