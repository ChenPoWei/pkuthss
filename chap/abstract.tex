% Copyright (c) 2014,2016 Casper Ti. Vector 
 
\begin{cabstract}

	⽐特币是⼀个集成网络学、密码学、货币银⾏学并以区块链为基础的加密货币。有别于其他加密货币的是⽐特币特有的去中⼼化与匿名化。区块链技术虽能保有交易信息的透明性与交易数据的不可变动性,但也因为其匿名之特性存在着三项问题。第一,在⽐特币交易系统中,没有落实实名制,造成流动资金的不透明性。其次,现今的国家并无⽀持以⽐特币交易相关的⽀付系统或是制定出相关的税务标准,使得政府无法从加密货币这⽅⾯的⾦融交易获得税收。第三,现有的⽐特币交易模型皆为匿名与匿名之间的交易,并无开⽴交易凭据,亦无法保障消费者权益。

	本论⽂设计与实现⼀个比特币的实时交易监督系统-BRTMS,论文工作包括分析五种交易模型,发现由匿名顾客支付给实名商家的交易监督可同时实践保障消费者权益、保护顾客隐私以及使政府可以课征税收。由需求分析将系统模块划分为用户注册与登入、产品管理、职工管理、商家交易管理,以及顾客交易管理五种模块,同时采用Java编程语言和比特币开源钱包,以五种模块为基础实现主要系统,以及商家和商品信息管理、商家手持移动装置收款及交易、客户端行动支付和交易三个子系统,并且为解决区块链交易速度缓慢问题引入多重签章算法构建实时监督系统。实现系统后,对本系统进行功能测试,且对原始监督系统与有多重签章算法支持的实时监督系统进行性能测试与分析。比特币通过本系统的实时交易监督,当顾客使用手持移动装置客⼾端可拥有详细的交易明细,也保障了消费者权益;借由商家和商品信息管理子系统,让商家可以容易进⾏商品管理及库存管理;在政府端实现多重签章算法,让商家可避免双重⽀付攻击,且大幅减少交易确认时间,更能使政府主管机关有效获得税收。

\end{cabstract}

\begin{eabstract}
	Bitcoin is a blockchain-based cryptocurrency that integrates networking, cryptography, and money banking. Unlike other cryptocurrencies, Bitcoin has unique decentralization and anonymization. Its applied blockchain technology can preserve the transparency of transaction information and the irreversibility of transaction data, but it also has three problems because of its anonymous nature. First, identifying users in Bitcoin trading has not been implemented. Therefore, it results in the opaqueness of transaction liquidity. Second, current countries don't support the cryptocurrency payment systems for the Bitcoin, or formulate relevant tax standards, making it impossible for the government to obtain tax revenues from cryptocurrency transactions. Third, the existing Bitcoin transaction models are all based on anonymity. There are no receipts and even the protection of consumer right.

	This thesis designs and implements a Bitcoin real-time transaction monitoring system-BRTMS. The research activities of this thesis include analysis of five trading models , finding the supervision of transactions between the anonymous customers and the real-name businesses, in that way can simultaneously protect consumer right, customer privacy, and enable the government to collect taxes. From the system requirements analysis, the system is divided into five modules: user registration and login, product management, employee management, merchant transaction management, and customer transaction management. Besides, the Java programming language and Bitcoin open source wallet are applied to realize the main system based on the five modules, as well as the three subsystems including store and merchandise information management subsystem ,store mobile payment collection and transaction subsystem, client mobile payment and transaction subsystem. Furthermore, to solve the problem of slow blockchain transactions, multiple signatures algorithm is introduced to build the real-time monitoring system. After the implement of the system, the functional testing is conducted. Then, the performance testing and analysis between the original supervisory system and the real-time supervisory system with multiple signatures algorithm are performed. With these evaluation results from the system, Bitcoin can let the customers know  more detailed information of transaction when using the mobile device subsystem. It can also protect the consumer right. Meanwhile, the store mobile payment collection and transaction subsystem can help businesses easier to carry out commodity management and inventory management. Implementing multiple signatures algorithm on the government side prevents businesses from double-spending attacks and drastically reduces the time required in transactions. Finally, the system can even enables government agencies to effectively obtain tax revenues.


\end{eabstract}

% vim:ts=4:sw=4
