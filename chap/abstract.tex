% Copyright (c) 2014,2016 Casper Ti. Vector
% Public domain.
% 摘要
\begin{cabstract}

	金融科技蓬勃發展的今天,區塊鏈技術也是重點發展對象。區塊鏈技術最著名的代表作,不外乎是於2009年Satoshi Nakamoto提出的一篇名為比特幣:一種點對點式的電子現金系統(Bitcoin: A Peer-to-Peer Electronic Cash System)論文\parencite{bitcoinpaper},奠定了區塊鏈技術的開始,以及與貨幣銀行學緊密的結合。比特幣是一個集成網絡學、密碼學、貨幣銀行學的加密貨幣,現今的加密貨幣市場中,有數以千計的貨幣種類在市場流動著。值得一提的是,於2009年開始運作至今(2018年),比特幣點對點式的電子現金系統還未出現過錯誤,這也體現出比特幣可以承受將近十年來各式各樣的網絡攻擊以及在程序上並無太⼤漏洞瑕疵等優點。比特幣最大的特色在於去中心化與匿名化,以去中⼼化的基礎建構出⼀個政府無法管控的點對點⾦流,但也因為其匿名之特性,使得政府相關人⼠難以追查每⼀筆資⾦的真正持有者是誰,在傳統的中心化銀行跨國轉帳中都需要基本的實名制驗證,藉由實名制有效過濾洗錢的發生。但在比特幣點對點的電子現金系統中,沒有任何一個使用者或是政府可以要求每一個人落實實名制,進而增加交易追蹤、洗錢防制的困難性。除了不可管控、難以追蹤的特點外,在國家政府方面稅收更是國家基礎運作的資⾦來源,因為現今的國家並無支持比特幣相關的收銀系統或是制定出相關的稅務標準,也使得國家政府無法從這方面獲得稅收資金。

	經由深度瞭解比特幣的運作原理,再由上述無法管理資金流、無法追蹤、無法得到稅收三項出發點,本論文致力於設計一個区块链的实名交易监督系统。在設計該系統前,也探討了多種模式下的交易模型,發現現金支付已經存在匿名支付給匿名、匿名支付給實名的模型,在刷卡支付中有著實名支付給實名、實名支付給匿名再支付給實名,上述的四種模型。經由上述模式的分析,可以得知,個人隱私的意識崛起,唯由匿名支付給實名時,才可以做到不透露消費者信息,亦可做到保障消費者權益。在點對點的電子現金市場中,還是停留在匿名支付給匿名的模式中,本論⽂致⼒於設計出匿名⽀付實名的加密貨幣市場之監督收銀系統,以實踐消費者匿名,同時也讓消費者擁有費者權益的交易模型。
\end{cabstract}

\begin{eabstract}
	Financial technology is booming today, the applied blockchain technology is also the focus of development. The most famous masterpiece of blockchain technology is just from a paper by Satoshi Nakamoto in 2009 titled Bitcoin: A Peer-to-Peer Electronic Cash System. It is the cornerstone of blockchain technology and then tightly integrated with currency banking. Bitcoin is a cryptocurrency that integrates network science, cryptography, and currency banking. In today's cryptocurrency market, thousands of currencies are circulating in markets. It is worth mentioning that, it started operation in 2009 so far (2018). Bitcoin's peer-to-peer electronic cash system has not seen any mistakes yet. This also shows that Bitcoin can withstand a wide range of network attacks in the past decade and is not flawed in the procedure. The biggest feature of Bitcoin is decentralization and anonymization. Building on the democratization foundation is a peer-to-peer flow that the government can not control because bitcoin has anonymity. Making it difficult for government stakeholders to trace down the true owners of each funding package. In the traditional central bank transnational transfer needs basic real-name verification, with the real-name system to effectively filter the occurrence of money laundering. But in Bitcoin's peer-to-peer e-cash system, no single user or government can require everyone to implement a real-name system. Bitcoin, in turn, increases transaction tracking and the prevention of money laundering. Except for the unmanageable and hard-to-trace characteristics, the tax revenue of the state government is the source of funding for the operation of the state. Because today's countries do not support Bitcoin-related cash register system or set the relevant tax standards, but also makes the national government can not get tax revenue in this area. 


	Through the depth understanding of the working principle of bitcoin. Again by the above cannot manage the flow of funds, cannot be tracked, cannot get tax three starting point. This thesis is devoted to designing a "blockchain real-name transaction monitoring system". Before designing the system, several trading models were also explored. Found that there are cash payments to pay anonymous anonymous, anonymous payment to real-name model, in the credit card payment has a real name payment to the real name, real name paid to anonymous and then paid to the real name, the above four models. Through the analysis of the above model can be learned. The rise of the awareness of personal privacy, only paid by the anonymous real name, we can do without revealing consumer information, consumer protection can also be done to protect rights and interests. In peer-to-peer electronic cash, it still stays in the anonymous mode of payment anonymously. This essay aims to design a supervisory cashier system for anonymous real-name cryptocurrencies. To practice the consumer anonymity, but also allow consumers to have the rights and interests of the transaction model. 

\end{eabstract}

% vim:ts=4:sw=4
