% Copyright (c) 2014,2016 Casper Ti. Vector
% Public domain.

\begin{cabstract}
	%\pkuthssffaq % 中文测试文字
	金融科技蓬勃發展的今天,區塊鏈技術也是重點發展對象。區塊鏈技術最著名的代表作,不外乎是於2009年中本聰提出的一篇名為比特幣:一種點對點式的電子現金系統》(Bitcoin: A Peer-to-Peer Electronic Cash System)論文\parencite{bitcoinpaper},奠定了區塊鏈技術的開始,以及於貨幣銀行學緊密的結合。比特幣是一個集成網路學、密碼學、金融學的密碼貨幣,現今的密碼貨幣市場中,有數以千計的貨幣種類在市場流動著。執得一提的是,於2009年開始運作至今(2018年),比特幣點對點式的電子現金系統,還未出現過錯誤,這也體現了比特幣可以承受將近十年來各式各樣的網路攻擊以及在程序上並無太大的漏洞瑕疵。比特幣最大的特色在於去中心化、匿名化,因為去中心化的基礎建構出一個政府無法管控的點對點的金流,也因為匿名,使得政府相關人士難你去追查每一筆資金的真正持有者是誰,在傳統的中心化銀行跨國轉帳中都需要基本的實名制驗證,藉由實名制有效過濾洗錢的發生。但在彼特幣點對點的電子現金系統中,沒有任何一個使用者或是政府可以要求每一個人的實名制,促使的交易追蹤、洗錢防制變得更加的困難。除了不可管控、難以追蹤的特點外,在國家政府方面稅收更是國家繼續運作的基礎資金來源,因為現今的國家並無支持比特幣相關的收銀機或是制定出相關的標準稅務,也使得國家政府無法在這方面獲得稅務資金。

	經由深度的了解比特幣的運作原理,再由上述無法管理資金流、無法追蹤、無法得到稅收,三項出發點,本論文致力於設計一個比特幣的收銀監督系統。在設計該系統前,也探討了多種場景下的交易模型,發現現金已經存在匿名支付給匿名、匿名支付給實名的模型,在刷卡支付中有著實名支付給實名、實名支付給匿名再支付給實名,上述的四種模型。經有上述的分析,可以得知,個人隱私的意識崛起,唯有匿名支付給實名時,才可以做到不透露消費者信息,亦可做到消費者權益的申訴權。在點對點的電子現金的市場中,還是停留在匿名支付給匿名的場景中,本論文致力於設計一個匿名支付實名的密碼貨幣市場的監督收銀系統,以實踐消費者匿名,同時也讓消費者擁有費者權益的交易模型。
\end{cabstract}

\begin{eabstract}
	Test of the English abstract.
\end{eabstract}

% vim:ts=4:sw=4
