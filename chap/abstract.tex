% Copyright (c) 2014,2016 Casper Ti. Vector

\begin{cabstract}

	⽐特幣是⼀個集成網絡學、密碼學、貨幣銀⾏學並以區塊鏈為基礎的加密貨幣。有別於其他加密貨幣的是⽐特幣特有的去中⼼化與匿名化。區塊鏈技術雖能保有交易信息的透明性與交易數據的不可變動性,但也因為其匿名之特性存在著三項問題。第一,在⽐特幣交易系統中,沒有落實實名制,造成流動資金的不透明性。其次,現今的國家並無⽀持以⽐特幣交易相關的⽀付系統或是制定出相關的稅務標準,使得政府無法從加密貨幣這⽅⾯的⾦融交易獲得稅收。第三,現有的⽐特幣交易模型皆為匿名與匿名之間的交易,並無開⽴收據,亦無法保障消費者權益。

	本論⽂設計與實現⼀個比特幣的實時交易監督系統-BRTMS,論文工作包括分析五種交易模型,發現由匿名顧客支付給實名商家的交易監督可同時實踐保障消費者權益、保護顧客隱私以及使政府可以課徵稅收。由需求分析將系統模塊劃分為用戶註冊與登入、產品管理、職工管理、商家交易管理,以及顧客交易管理五種模塊,同時採用Java編程語言和比特幣開源錢包,以五種模塊為基礎實現主要系統,以及商家和商品信息管理、移動裝置收款及交易、客戶端移動支付和交易三個子系統,並且為解決區塊鏈交易速度緩慢問題引入多重簽章算法構建實時監督系統。實現系統後,對本系統進行功能測試,且對原始監督系統與有多重簽章演算法支援的實時監督系統進行性能測試與分析。比特幣通過本系統,當顧客使用移動裝置客⼾端擁有有詳細的交易明細,同時保障消費者權益;藉由商家和商品信息管理子系統,讓商家可以更容易進⾏商品管理及庫存管理;在政府端實現多重簽章算法,讓商家可預防雙重⽀付攻擊,且大幅減少平均交易所需時間,也能使政府主管機關有效獲得稅收。



\end{cabstract}

\begin{eabstract}
	Bitcoin is a blockchain-based cryptocurrency that integrates cybernetics, cryptography, and money banking. Unlike other cryptocurrencies, Bitcoin's unique decentralization and anonymization. Although blockchain technology can preserve the transparency of transaction information and the invariability of transaction data, it also has three problems because of its anonymous nature. First, in the Bitcoin trading system, the real name system has not been implemented, resulting in the opaqueness of liquidity. Second, today's countries do not support the use of Bitcoin to trade-related payment systems or formulate relevant tax standards, making it impossible for the government to obtain tax revenues from cryptocurrency financial transactions. Thirdly, the existing Bitcoin transaction models are all transactions between anonymous and anonymous transactions. There are no receipts and consumers' rights cannot be protected.


	This thesis designs and implements a Bitcoin real-time transaction monitoring system-BRTMS. Thesis work includes analysis of five trading models. It is found that the supervision of transactions paid by anonymous customers to real-name businesses can simultaneously protect consumer rights, protect customer privacy, and enable the government to collect taxes. From the demand analysis, the system modules are divided into five modules: user registration and login, product management, employee management, merchant transaction management, and customer transaction management. At the same time, the Java programming language and Bitcoin open source purse are used to implement the main system based on the five modules, as well as the three subsystems of store and merchandise information management subsystem, mobile device collection and client mobile payment and transaction subsystem.And to solve the problem of slow blockchain transactions, multiple signature algorithms are introduced to build a real-time monitoring system.After the system is implemented, the system is functionally tested, and performance testing and analysis are performed on the original supervisory system and the real-time supervisory system that supports multiple signature algorithms.Bitcoin through this system, when customers use the mobile device client has a detailed transaction details, while protecting the rights of consumers; through the merchant and commodity information management subsystem, so that businesses can more easily carry out commodity management and inventory management; Implementing multiple signature algorithms on the government side allows businesses to prevent double-payment attacks and drastically reduce the time required for average transactions. It also enables government agencies to effectively obtain tax revenues.

\end{eabstract}

% vim:ts=4:sw=4
