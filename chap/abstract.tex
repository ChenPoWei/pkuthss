% Copyright (c) 2014,2016 Casper Ti. Vector

\begin{cabstract}

	⽐特幣是⼀個集成網絡學、密碼學、貨幣銀⾏學並以區塊鏈為基礎的加密貨幣。有別於其他加密貨幣的是⽐特幣特有的去中⼼化與匿名化。區塊鏈技術雖能保有交易信息的透明性與交易數據的不可變動性,但也因為其匿名之特性存在著三項問題。第一,在⽐特幣交易系統中,沒有落實實名制,造成流動資金的不透明性。其次,現今的國家並無⽀持以⽐特幣交易相關的⽀付系統或是制定出相關的稅務標準,使得政府無法從加密貨幣這⽅⾯的⾦融交易獲得稅收。第三,現有的⽐特幣交易模型皆為匿名與匿名之間的交易,並無開⽴交易憑據,亦無法保障消費者權益。

	本論⽂設計與實現⼀個比特幣的實時交易監督系統-BRTMS,論文工作包括分析五種交易模型,發現由匿名顧客支付給實名商家的交易監督可同時實踐保障消費者權益、保護顧客隱私以及使政府可以課徵稅收。由需求分析將系統模塊劃分為用戶註冊與登入、產品管理、職工管理、商家交易管理,以及顧客交易管理五種模塊,同時採用Java編程語言和比特幣開源錢包,以五種模塊為基礎實現主要系統,以及商家和商品信息管理、移動裝置收款及交易、客戶端行動支付和交易三個子系統,並且為解決區塊鏈交易速度緩慢問題引入多重簽章算法構建實時監督系統。實現系統後,對本系統進行功能測試,且對原始監督系統與有多重簽章演算法支援的實時監督系統進行性能測試與分析。比特幣通過本系統的實時交易監督,當顧客使用移動裝置客⼾端可擁有詳細的交易明細,也保障了消費者權益;藉由商家和商品信息管理子系統,讓商家可以容易進⾏商品管理及庫存管理;在政府端實現多重簽章算法,讓商家可避免雙重⽀付攻擊,且大幅減少交易確認時間,更能使政府主管機關有效獲得稅收。

\end{cabstract}

\begin{eabstract}
	Bitcoin is a blockchain-based cryptocurrency that integrates networking, cryptography, and money banking. Unlike other cryptocurrencies, Bitcoin has unique decentralization and anonymization. Its applied blockchain technology can preserve the transparency of transaction information and the irreversibility of transaction data, but it also has three problems because of its anonymous nature. First, Identifying users in Bitcoin trading has not been implemented. Therefore, it results in the opaqueness of transaction liquidity. Second, current countries don't support the cryptocurrency payment systems for the Bitcoin, or formulate relevant tax standards, making it impossible for the government to obtain tax revenues from cryptocurrency transactions. Third, the existing Bitcoin transaction models are all based on anonymity. There are no receipts and even the protection of consumer right.

	This thesis designs and implements a Bitcoin real-time transaction monitoring system-BRTMS. The research activities of this thesis include analysis of five trading models , finding the supervision of transactions between the anonymous customers and the real-name businesses, in that way can simultaneously protect consumer right, customer privacy, and enable the government to collect taxes. From the system requirements analysis, the system is divided into five modules: user registration and login, product management, employee management, merchant transaction management, and customer transaction management. Besides, the Java programming language and Bitcoin open source wallet are applied to realize the main system based on the five modules, as well as the three subsystems including store and merchandise information management subsystem ,store mobile payment collection and transaction subsystem, client mobile payment and transaction subsystem. Furthermore, to solve the problem of slow blockchain transactions, multiple signatures algorithm is introduced to build the real-time monitoring system. After the implement of the system, the functional testing is conducted. Then, the performance testing and analysis between the original supervisory system and the real-time supervisory system with multiple signatures algorithm are performed. With these evaluation results from the system, Bitcoin can let the customers know  more detailed information of transaction when using the mobile device subsystem. It can also protect the consumer right. Meanwhile, the store mobile payment collection and transaction subsystem can help businesses easier to carry out commodity management and inventory management; Implementing multiple signatures algorithm on the government side prevents businesses from double-spending attacks and drastically reduces the time required in transactions. Finally, the system can even enables government agencies to effectively obtain tax revenues.


\end{eabstract}

% vim:ts=4:sw=4
