% Copyright (c) 2014,2016 Casper Ti. Vector
% Public domain.

\chapter{多重簽章優化比特幣交易監督系統}
%\pkuthssffaq % 中文测试文字。
本節將詳述本系統之商家註冊、綠色地址錢包創建與驗證交易的運作流程與相關資料庫架構 。

	\paragraph{第一步}商家以通過政府機構的審查稽核的帳戶登入該系統。
	\paragraph{第二步}系統載入該店家註冊的商品資訊,店員可以依照客戶的需求進行點單選取數量。
	\paragraph{第三步}快速建立交易清單,並透過Green address Testnet建立一個全新的比特幣收款地址,再以Android Beam[23]的方式將交易資訊輕鬆地傳得給顧客。
	\paragraph{第四步}在商家店員的平板電腦收到這筆交易信息之後,會對本監督系統重送一個副本進行存檔。該交易資訊包括由監督系統所提出的交易流水號、商家編號等資訊。
	\paragraph{第五步}消費者收到交易信息後,手機會自動開啟Green ad-dress的付款頁面,確認金額無誤之後便能進行支付,此時便會以客戶的比特幣私鑰簽署交易,並等待綠色地址代理節點的認證及發布。
	\paragraph{第六步}綠色地址代理節點收到交易請求,並完成驗證非雙花攻擊後,以代理節點對應地址的私鑰簽署本次交易,並廣播至比特幣節點中。
	\paragraph{第七步}區塊鏈檢視器便會開始分析網路中存在的所有在緩存池中的交易,以及已經被記錄到區塊鏈中的交易。
	\paragraph{第八步}本交易監督系統會向區塊鏈檢視器查找,檢查該筆交易是否已經存在於緩存池當中,若已經確認進入緩存池,則認定該筆交易成立並完成付款。
	\paragraph{第九步}在交易確認之後,便向商家店員的平板電腦送出交易已經成交的信息,此時完成交易,與此同時也將一筆交易資訊建置於系統資料庫內。
