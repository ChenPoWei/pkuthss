% Copyright (c) 2014,2016 Casper Ti. Vector
% Public domain.

\chapter{以多重簽章優化區塊鏈的實名交易監督系統之實作} 

在這樣的機制下,雖然Green Address 沒有減少交易確認時間,但是只要是用Green Address 即可確保雙重支付攻擊是不會發生的,對商家或是收款人而言,可以得到在即時交易中不被雙重支付攻擊的保障,提升未進入區塊鏈的交易可信度,進而創造出即時交易的可行性。 

本節將詳述本系統之商家註冊、Green Address錢包創建與驗證交易的運作流程與相關數據庫架構,如圖\ref{gabpcss}所示。

	\begin{figure}[h]
		\centering
		\includegraphics[width = 0.8\textwidth]{gabpcss.png}
		\caption{多重簽章優化後的BRTMS整體示意圖}\label{gabpcss}
	\end{figure}

	\begin{enumerate}
		\item 商家以通過政府機構的審查稽覈的帳戶登入該系統。
		\item 系統載入該店家註冊的商品信息,店員可以依照客戶的需求進行點單選取數量。
		\item 快速建立交易清單,並透過Green Address建立一個全新的比特幣收款地址,再以Android Beam的方式將交易信息輕鬆地傳達給消費者。
		\item 在商家店員的平板電腦收到這筆交易信息之後,會對本監督系統重送一個副本進行存檔。該交易信息包括由監督系統所提出的交易流水號、商家編號等信息。
		\item 消費者收到交易信息後,手機會自動開啟Green Address的付款頁面,確認金額無誤之後便能進行支付,此時便會以客戶的比特幣私鑰簽署交易,並等待Green Address機構節點的認證及發布。
		\item Green Address機構節點收到交易請求,並完成驗證非雙重支付攻擊後,以代理節點對應地址的私鑰簽署本次交易,並廣播至比特幣節點中。
		\item 區塊鏈檢視器便會開始分析網絡中所有存在緩存池中的交易,以及已經被記錄到區塊鏈中的交易。
		\item 本交易監督系統會向區塊鏈檢視器查找,檢查該筆交易是否已經存在於緩存池當中,若已經確認進入緩存池,則認定該筆交易成立並完成付款。
		\item 在交易確認之後,便向商家店員的平板電腦送出交易已經成交的信息,此時完成交易,於此同時也將該筆交易信息建置於系統數據庫內。
	\end{enumerate}