% Copyright (c) 2014,2016 Casper Ti. Vector
% Public domain.

\chapter{多重簽章優化区块链的实名交易监督系统}

比特幣區塊鏈技術,雖然已經利用工作量證明的方式解決了雙重支付問題,但工作量證明的演算法所設定之題目困難度會直接影響到每一個比特幣區塊的產出時間,這個比特幣區塊的產出時間也考慮到比特幣全節點於全世界各地的網路同步狀況,倘若今天的區塊生成時間過短會造成全世界的比特幣節點之區塊數據不一致,這樣的數據不一致將導致比特幣區塊鏈出現分岔,在更嚴重一點甚至會造成比特幣網路的瓦解。 

何謂雙重支付問題,雙重支付問題存在於比特幣交易在未被區塊鏈確認收錄到區塊鏈之前,都有機會受到惡意的攻擊者重複消費同一筆金額,而這些雙重支付交易會存在於交易緩存池當中,雙重支付的交易會在記載到區塊中的同時被過濾。現今的比特幣區塊產出速度為十分鐘一塊,但十分鐘的確認時間對實體店面的小額交易事件會造成很大的不友善,為了在既有的比特幣區塊鏈框架底下能夠提升交易速度,因此Green Address 技術致力於在一開始創建交易的同時管控雙重支付交易的發生,Green Address的技術採用了多重簽章2-of-2,2-of-2的意思是創建一個特殊的比特幣地址,這個比特幣地址的持有人有兩個人,其一為使用者,另一位則為Green Address機構節點,這筆交易的建立必須要雙方同時簽署才得以被成立。Green Address 機構節點也就成為了交易創建的把關者,過濾所有雙重支付攻擊的發生。 

在這樣的機制下,雖然Green Address 沒有減少交易確認時間,但是只要是用Green Address 即可確保雙重支付攻擊是不會發生的,對商家或是收款人而言,可以得到在即時交易中不被雙重支付攻擊的保障,提升未進入區塊鏈的交易可信度,進而創造出即時交易的可行性。 

本節將詳述本系統之商家註冊、Green Address錢包創建與驗證交易的運作流程與相關數據庫架構 。

	\begin{figure}[h]
		\centering
		\includegraphics[width = 0.8\textwidth]{gabpcss.png}
		\caption{gabpcss}\label{gabpcss}
	\end{figure}

	\begin{enumerate}
		\item 商家以通過政府機構的審查稽核的帳戶登入該系統。
		\item 系統載入該店家註冊的商品資訊,店員可以依照客戶的需求進行點單選取數量。
		\item 快速建立交易清單,並透過Green Address Testnet建立一個全新的比特幣收款地址,再以Android Beam[23]的方式將交易資訊輕鬆地傳達給消費者。
		\item 在商家店員的平板電腦收到這筆交易信息之後,會對本監督系統重送一個副本進行存檔。該交易資訊包括由監督系統所提出的交易流水號、商家編號等資訊。
		\item 消費者收到交易信息後,手機會自動開啟Green Address的付款頁面,確認金額無誤之後便能進行支付,此時便會以客戶的比特幣私鑰簽署交易,並等待Green Address機構節點的認證及發布。
		\item Green Address機構節點收到交易請求,並完成驗證非雙重支付攻擊後,以代理節點對應地址的私鑰簽署本次交易,並廣播至比特幣節點中。
		\item 區塊鏈檢視器便會開始分析網路中存在的所有在緩存池中的交易,以及已經被記錄到區塊鏈中的交易。
		\item 本交易監督系統會向區塊鏈檢視器查找,檢查該筆交易是否已經存在於緩存池當中,若已經確認進入緩存池,則認定該筆交易成立並完成付款。
		\item 在交易確認之後,便向商家店員的平板電腦送出交易已經成交的信息,此時完成交易,與此同時也將一筆交易資訊建置於系統數據庫內。
	\end{enumerate}